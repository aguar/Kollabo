\documentclass{article}
\usepackage[ngerman]{babel} 
\usepackage{amsmath}
\usepackage[utf8]{inputenc}
\begin{document}

\title{CS102 \LaTeX  \, Übung}
\author{Andrea Guarneri}
\date{\today}
\maketitle

\section{Erster Abschnitt}
Hier könnte Ihre Werbung stehen.

\section{Tabelle}
Hier meine erste Tabelle in \LaTeX : \\
\begin{table}[h]
\centering
\begin{tabular}{c|c|c|c}
& Punkte erhalten & Punkte möglich & \% \\
 \hline Aufgabe 1 & 2 & 4 & 0.5 \\
 Aufgabe 2 & 3 & 3 & 1 \\
 Aufgabe 3 & 3 & 3 & 1 \\
 \end{tabular}
 
\caption{Diese Tabelle könnte ebenfalls Ihre Werbung enthalten.}
\end{table}

\section{Formeln}
\subsection{Pythagoras}
Der Satz des Pythagoras errechnet sich wie folgt: $a^2 + b^2 = c^2 $. Daraus können wir die Länge der Hypothenuse wie folgt berechnen: $c = \sqrt{a^2+b^2}$

\subsection{Summen}
Wir können auch die Formel für eine Summe angeben:
\begin{equation}
s = \sum_{i=1}^n i\ = \frac{n*(n+1)}{2}
\end{equation}
\section{Übung 7}
Ich war hier - Andrea Guarneri

\end{document} 